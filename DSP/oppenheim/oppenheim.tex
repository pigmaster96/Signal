\documentclass{report} 
\title{Signals And Systems by Alan V. Oppenheim:\\Notes}
\date{Started 17 April 2025}
\author{Malcolm}
\usepackage{amsmath} %import math
\usepackage{mathtools} %more math
\usepackage{amssymb} %for QED symbol
\usepackage{amsthm} %
\usepackage{bm}%bold math
\usepackage{graphicx} %import imaging
\graphicspath{{./images/}} %set imaging path
\begin{document}
\maketitle

\tableofcontents

\newpage
\section{Introduction}
\subsection{Signal Energy and Power}
\textbf{Motivation and definition}\\
In many but not all, applications, the signals considered directly related to physical quantities capturing
power and energy in a physical system. (for instance $v^2/R$ for the power across a resistor)\\
\vspace{1mm}\\
As such it is a common and worthwhile convention to use similar terminology for power and energy for \textit{any} 
continuous-time signal, denoted $x(t)$, or any discrete-time signal $x[n]$. 
In this case, the total energy over the time interval $t_1\leq t\leq t_2$ in a continuous signal $x(t)$ is defined
as
\begin{equation*}
\int^{t_2}_{t_1}|x(t)|^2dt
\end{equation*}
where $|x|$ denotes the magnitude of the (possibly complex) number $x$; see that the time-averaged signal 
can be obtained by dividing by $(t_2-t_1)$. Similarly for a discrete signal $x[n]$ over the interval $n_1\leq n\leq n_2$ the total energy is
\begin{equation*}
\sum^{n_2}_{n=n_1}|x[n]|^2
\end{equation*}
with the average power calculated by dividing by $(n_2-n_1+1)$.\\
\vspace{1mm}\\
It is important to remember that the terms `power' and `energy' are used here \textit{independently} of their 
relation to physical energy (they clearly don't correlate since their units or scalings would differ). Nevertheless
we will find it convenient to use these terms in a general fashion.\\
\vspace{1mm}\\
\textbf{Power and energy over infinite intervals}\\
Considering signals over an infinite time interval, meaning for $-\infty<t<+\infty$ or $-\infty<n<+\infty$. 
Here we define the total energy as the limits of the aforementioned equations increase without bound; in continuous
time,
\begin{equation*}
E_\infty\triangleq\lim_{T\to\infty}\int^T_{-T}|x(t)|^2dt
=\int^{+\infty}_{-\infty}|x(t)|^2dt
\end{equation*}
and in discrete time,
\begin{equation*}
E_\infty\triangleq\lim_{N\to\infty}\sum^{+N}_{n=-N}|x[n]|^2=\sum^{+\infty}_{n=-\infty}|x[n]|^2
\end{equation*}
Note that these expressions may not converge; for instance say $x(t)$ or $x[n]$ equal some nonzero constant for 
all time: such signals have infinite energy, while signals with $E_\infty<\infty$ have finite energy.\\
(next page)\newpage
\noindent\textbf{Cont.}\\
Analagously, we can define the time-averaged power over an infinite interval as
\begin{equation*}
P_\infty\triangleq\lim_{T\to\infty}\frac{1}{2T}\int^T_{-T}|x(t)|^2dt
\end{equation*}
and
\begin{equation*}
P_\infty\triangleq\lim_{N\to\infty}\frac{1}{2N+1}\sum^{+N}_{n=-N}|x[n]|^2
\end{equation*}
In continuous and discrete time respectively. \\
\vspace{1mm}\\
\textbf{Intuition}\\
See that with these definitions, we can identify three classes of signals: first those with finite total energy,
meaning $E_\infty<\infty$. See that such a signal would have zero average power:
\begin{equation*}
P_\infty=\lim_{T\to\infty}\frac{E_\infty}{2T}=0
\end{equation*}
Second would be signals with finite average power $P_\infty$; see from the above expression that for $P_\infty>0$,
this requires that $E_\infty=\infty$.\\
\vspace{1mm}\\
Last would be signals for which neither $P_\infty$ nor $E_\infty$ are finite. An example of this might be $x(t)=t$.










\end{document}
